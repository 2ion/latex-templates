% vim: tw=0
\documentclass[ version=last,
                backaddress=on,
                12pt,
                a4paper,
                firstfoot=false,
                fromrule=false,
                fromlogo=false,
                enlargefirstpage=true]{scrlttr2}

\usepackage{relsize,xspace,ulem}
\usepackage{fontspec}
\usepackage{xunicode}
\usepackage{xltxtra}
\usepackage{tabularx,booktabs,array,dcolumn}
\usepackage[hang,flushmargin]{footmisc}
\newcommand{\otoprule}{\midrule[\heavyrulewidth]}
\newcolumntype{+}{>{\global\let\currentrowstyle\relax}}
\newcolumntype{^}{>{\currentrowstyle}}
\newcommand{\rowstyle}[1]{\gdef\currentrowstyle{#1}%
#1\ignorespaces
}
\newcolumntype{Z}{>{\raggedright\let\newline\\\arraybackslash}X}

\newfontfeature{Microtype}{protrusion=default;expansion=default;}
\setromanfont[Ligatures={Common,TeX},
    Numbers={Proportional},
    Kerning={On}, Microtype]{Minion Pro}
\setsansfont[Ligatures={Common,TeX},
    BoldFont={* Semibold},
    Numbers={Proportional},
    Kerning={On},Microtype,Scale=0.95]{Myriad Pro}
\setmonofont{Source Code Pro}
\newfontfamily{\cjkfontfamily}{Kozuka Gothic Pro}
\DeclareTextFontCommand{\textcjk}{\cjkfontfamily}

\usepackage{polyglossia}
\setdefaultlanguage{german}

\usepackage{csquotes}

\setkomavar{fromaddress}{
    Straße 6\\
    00000 Ort
}

\setkomavar{fromemail}{from@nowhere.de}
\setkomavar{fromname}{Max Muster}
\setkomavar{backaddress}{Max Muster, Straße 6, 00000 Ort}
\setkomavar{fromphone}{+49 (0) 1111 2222 3333}
\setkomavar{fromzipcode}{00000}
\setkomavar{place}{Ort}
\KOMAoptions{parskip=half-}

\begin{document}

\pagestyle{headings}

\setlength{\tabcolsep}{12pt}

\setkomavar{subject}{\sffamily Betreffzeile(n)}

\begin{letter}[fromalign=right,fromphone,fromemail]{
An den\\
Bundesminister der Finanzen\\
Herrn Olaf Scholz\\
Wilhelmstraße 97\\
10117 Ort
}

\opening{Sehr geehrter Herr Scholz,}

% source: https://www.dsw-info.de/steuerirrsinn/

ich schreibe Ihnen heute, nachdem Sie in den letzten Wochen wiederholt über
    vorgeschlagene oder auch bereits beschlossene Maßnahmen die
    eigenverantwortliche Altersvorsorge und das langfristige Anlegen in
    Wertpapieren mit verschiedenen Entscheidungen belastet und torpediert
    haben.

Die Bundesregierung und auch Ihr Haus haben bisher keinerlei Maßnahmen
    ergriffen, um die Bundesbürger flächendeckend in Finanzfragen oder
    hinsichtlich ihrer Altersvorsorge so aufzuklären und zu unterstützen, dass
    ein selbstorganisierter und nachhaltiger Aufbau einer privaten finanziellen
    Vorsorge in Deutschland ermöglicht oder gar gefördert wird.

Stattdessen haben Sie durch ganz aktuelle und gezielte Entscheidungen und
    Vorschläge genau gegensätzliche Impulse gesetzt.

So ist für mich nicht nachvollziehbar, warum Sie ausgerechnet 10 Millionen
    Bundesbürger, die ihr bereits versteuertes Geld in Wertpapieren anlegen,
    durch eine Finanztransaktionssteuer, durch die gezielte Beibehaltung des
    Solidaritätszuschlages und nun auch noch durch die Versagung der
    steuerlichen Anrechnung von Totalverlusten beschränken und beschneiden
    wollen.

Damit unterdrücken und belasten Sie die eigenverantwortliche Altersvorsorge,
    was aufgrund der voraussichtlich noch sehr lange anhaltenden
    Niedrigzinsphase fatale Folgen haben wird.

In Bezug auf die Finanztransaktionsteuer möchte ich ausdrücklich
    unterstreichen, dass ich der ursprünglichen Intention als Reaktion auf die
    Finanzkrise, über eine solche Steuer hochspekulative Finanzgeschäfte sowohl
    ordnungspolitisch als auch finanzpolitisch zu adressieren, prinzipiell
    positiv gegenüberstehe. 

Warum aber die von Ihnen vorgesehene FTT ausschließlich Aktionäre und damit
    Anleger, die der Industrie wichtiges Eigenkapital zur Verfügung stellen,
    erfasst werden sollen, verschließt sich mir vollkommen. Damit treffen Sie
    sowohl die falschen Anlageobjekte als auch die falschen Anleger.

Eine Transaktionssteuer wäre überhaupt nur dann sinnvoll, wenn sie sich in ein
    neues, ausgewogenes System einbettet, das ein langfristiges und der
    Volkswirtschaft dienendes Zurverfügungstellen von Kapital fördert, indem
    die langfristige Anlage und Investition entlastet wird. 

So ist auch die Frage zu stellen, warum der Verkauf von Immobilien nach zehn
    Jahren oder von Gold nach bereits einem Jahr steuerfrei möglich ist, bei
    Aktien als klassischem Sachwert aber nicht. Hier sollten Sie ansetzen und
    mindestens einen Gleichlauf herstellen, anstatt Aktionäre und Anleger
    gezielt mit neuen Steuern zu belasten.

Unabdingbar ist meiner Ansicht nach auch die deutliche Erhöhung des
    Sparerfreibetrages, um so auch einkommensschwächeren Bürgern und Familien
    zu ermöglichen, sich angemessen und eigenverantwortlich um ihre
    Altersvorsorge zu kümmern. 

Auch die isolierte Beibehaltung des Solidaritätszuschlages bei Kapitalerträgen
ist ein gutes Beispiel für die Diskriminierung von Anlegern. Die
verfassungsrechtliche Problematik Ihrer diesbezüglichen Vorschläge ist Ihnen
bewusst und es ist zudem höchst befremdlich, dass Sie damit ausgerechnet die
Bundesbürger treffen und belasten, die sich für ihre finanzielle Situation und
Altersvorsorge engagieren.

Ganz aktuell wollen Sie zudem die steuerliche Anrechenbarkeit aufgrund von
Totalausfällen erlittenen Verlusten ausschließen. Und auch hier treffen Sie
wieder private, langfristige Anleger von Aktien und Anleihen, die eben nicht
wie institutionelle Anleger schnell ein sinkendes Schiff verlassen. Mit Ihrem
Vorschlag zwingen Sie von Totalverlusten bedrohte Anleger, allein aus
fiskalischen Gründen Verluste zu realisieren, damit diese noch angerechnet
werden. Dies ist sowohl steuerlich als auch wirtschaftlich vollkommen absurd
und nicht im Ansatz nachvollziehbar. Auch der BFH vertritt in seinen
einschlägigen Entscheidungen die gegensätzliche Position. Sie fordern mit Ihrem
Vorstoß eine erneute gerichtliche Klärung heraus.

Alle drei Maßnahmen treffen normale Anleger - isoliert und in ihrer
Kombination. Vor diesem Hintergrund fordere ich Sie auf, Ihre Pläne dringend zu
überdenken und zu prüfen, ob Sie mit Ihren geplanten Maßnahmen
ordnungspolitisch und auch hinsichtlich Ihrer Verantwortung als Finanzminister
überhaupt die gewünschten Effekte erreichen können oder nicht vielmehr auch
rechtlich höchst umstrittene Belastungen und Signale setzen, die
Eigeninitiative zur Altersvorsorge bereits im Keim ersticken. Das kann und darf
eigentlich nicht in Ihrem Sinne sein.


\closing{Mit freundlichen Grüßen,}

\end{letter}

\end{document}
